\documentclass[12pt]{article}

% Language setting
\usepackage[utf8]{inputenc}
\usepackage[bulgarian]{babel}

% --------------------- Packages  --------------------
% Use biblatex
\usepackage{biblatex}
\addbibresource{bibliography.bib}
% Table thickness
\usepackage{ctable}
% Equations: SI units
\usepackage{siunitx}
% Approximately equal
\usepackage{amssymb}
% degrees symbol
\usepackage{gensymb}
% warning box
\usepackage{pifont,mdframed}
% Multiline math
\usepackage{amsmath}

\newenvironment{warning}
  {\par\begin{mdframed}[linewidth=2pt, linecolor=white]%
    \begin{list}{}{\leftmargin=1cm
                   \labelwidth=\leftmargin}\item[\Large\ding{43}]}
  {\end{list}\end{mdframed}\par}

% --------------------- Title  --------------------
\addbibresource{bibliography.bib}

\begin{document}

% Anfang der Titelseite________________________________________________________________________________
\begin{titlepage}
	\flushleft
	{\scshape\Large Протокол VII \hspace{2cm} Молекулна физика\par}
	\vspace{4cm}
	{\huge\bfseries Измерване на специфичен топлинен капацитет на метали с калориметър\par}
	\vspace{1cm}
	{\LARGE\bfseries Лабораторно упражнение №3.11\par}
	\vspace{5cm}
    {\LARGE\bfseries Виолета Кабаджова, \par}
    {\large\bfseries ККТФ, фак. номер: 3PH0600026\par}
	\vspace{1cm}
	
	{\large Физически Факултет, 
	
	Софийски Университет "Св. Климент Охридски"
	
    20 април 2023 г.\par}
	
\end{titlepage}

\section{Теоритична част}\label{sec:theoretical-part}
$C_V$ наричаме моларен топлинен капацитет при постоянен обем - количеството топлина, което трябва да обмени 1 $mol$ вещество при постоянен обем, за да се измени температурата му с един градус, $[C_V] = J/(mol\cdot K)$.

За единица маса се въвежда величината специфичен топлинен капацитет $c_V$ - количеството топлина, което трябва да обмени термодинамична система с маса 1 kg, за да се измени температурата ѝ с един градус, $[c] = J/(kg\cdot K)$.

В настоящото упражнение изследваме металите алуминий ($Al$), месинг ($CuZn$) и желязо ($Fe$), нагрявайки ги то тяхната пределна температура и след това охлаждайки ги в калориметър с цел откриване на техния специфичен топлинен капацитет.  

Изхождайки от формула \ref{eq:q1} за количество топлина, което нагретият образец ще отдаде към калориметричната система, и уравнение \ref{eq:q2} за количество топлина, което ще погълнат калориметърът и водата от нагретият образец, можем да изведем формула \ref{eq:work-formula}, определяща специфичния топлинен капацитет на различните образци. За температурите $T_K$ - температурата на кипене на водата в стъклената чаша, $T_M$ - равновесната температура и $T_1$ ($T_1 < T_M < T_K$) - температура на водата в калориметъра, ще бъде обяснено по-долу в експерименталната част.

\begin{equation}\label{eq:q1}
    Q_1 = c_om_o(T_K - T_M)
\end{equation}

\begin{equation}\label{eq:q2}
    Q_2 = c_BM_B(T_M - T_1) + C_K(T_M - T_1)
\end{equation}

\begin{equation}\label{eq:work-formula}
    c_o = \frac{(T_M-T_1)(c_BM_B + C_K)}{m_o(T_K - T_M)}
\end{equation}

\section{Експериментална част}\label{sec:experimental}
Експериментът се състои в нагряването на стъклена чаша с вода до температура $100 \degree C$. Образците на металите (малки кубчета от съответния метал) се потапят в кипящата вода до затоплянето им до възможно най-високата температура, която позволява съответния метал. След като се загреят до определената температура, образците се поставят в калориметъра, в който има поставена хладка вода. Течността в съда се разбърква постоянно до достигане на равновесна температура, която бива отчетена в края. 

\subsection{Задача 1: Измерване на специфичен топлинен капацитет на метали}
За определяне на специфичния топлинен капацитет използваме формула \ref{eq:work-formula}. С направата на действията, описани в началото на секция \ref{sec:experimental}, записваме отчетени стойности в таблица \ref{tbl:meas}. След всеки от експериментите сменяме водата в калориметъра и отчитаме масата ѝ наново. Използваните фиксирани величини (константи в рамките на експеримента) записваме в таблица \ref{tbl:params}. Абсолютната грешка в крайния резултат изчисляваме приблизително по формула \ref{eq:max-error}. За определяне на масата на водата първо измерваме съда на калориметъра, докато е празен, и след като сме го напълнили с вода; разликата между двете стойности представлява масата на водата.

\begin{equation}\label{eq:max-error}
    \frac{\Delta C}{C} = \frac{\Delta c_B}{c_B} + \frac{\Delta M_B}{M_B} + \frac{\Delta T_M}{T_M} + \frac{\Delta T_1}{T_1} + \frac{\Delta C_K}{C_K} + \frac{\Delta m_O}{m_O} 
\end{equation}

\begin{table}[h]
\begin{center}
\begin{tabular}{|l|l|l|l|} \hline
    величина & Алуминий (Al) & Месинг (CuZn) & Желязо (Fe) \\ \hline
    маса на образеца $m_o$, [g] & 60.1 & 119.6 & 120.2 \\ \hline
    маса на водата $M_B$, [g] & 236.9 & 242.8 & 258.7 \\ \hline
    максимална температура &&& \\ 
    на образеца $T_K$, [\degree C] & 96.8 & 96.6 & 95.1 \\ \hline
    равновесна температура $T_M$, [\degree C] & 27 & 26 & 27\\ \hline
    специфичен топлинен капацитет &&& \\
    на образеца $C$, [J/{kg.K}] & 1008 \pm 182 & 384 \pm 68 & 562 \pm 99 \\ \hline
\end{tabular}
\caption{\label{tbl:meas}Измервания за трите метала}
\end{center}
\end{table}

\begin{table}[h]
\begin{center}
\begin{tabular}{|l|l|} \hline
        топлинен капацитет на калориметъра $C_{K}$ & $66 \pm 1 J/K$ \\ \hline
        специфичен топлинен капацитет на водата $C_{B}$ & $4187 J/(g.K)$ \\ \hline
        първоначална температура & \\
        на водата в калориметъра $T_1$ & 23 $\degree C$\\ \hline
\end{tabular}
\caption{\label{tbl:params}Константни величини в рамките на експеримента}
\end{center}
\end{table}

\end{document}
